\documentclass[10pt]{article}
\usepackage[margin=0.6in]{geometry}
\usepackage{fancyhdr}
\usepackage{amsmath}
\usepackage{amssymb}
\usepackage{mathtools}
\usepackage{tikz}
\usepackage{multicol}
\usepackage{mathrsfs}
\usepackage{graphicx}
\pagestyle{fancy}



\begin{document}
\lhead{\color{blue}Algebra Pt.1 - Crib Sheet}
\chead{\color{blue}Henry Wise}
\rhead{\color{blue}Summer Exam EOY}

%The first year of algebra:
\begin{multicols}{2}
\begin{itemize}
%Complex numbers:
    \item $|z|=\sqrt{z\Bar{z}}$
    \item $\frac{\Bar{z}}{|z|^{2}}=\frac{x}{x^{2}+y^{2}}-\frac{y}{x^{2}+y^{2}}i\in\mathbb{C}$\\
    Proof: $z\times\frac{\Bar{z}}{|z|^{2}}=\frac{z\times\Bar{z}}{|z|^{2}}=1$
    \item $z+\Bar{z}=2\Re(z)$; $z-\Bar{z}=\Im(z)$; $\Bar{z+w}=\Bar{z}+\Bar{w}$; $\Bar{zw}=\Bar{z}\Bar{w}$
    \item $|\Bar{z}|=|z|$; $|zw|=|z||w|$; $|z/w|=|z|/|w|$; $\frac{1}{w}=\frac{\Bar{w}}{|w|^{2}}$\\
    Proof: $|\Bar{z}|^{2}=(|x-iy|)^{2}=x^{2}+(-y)^{2}=x^{2}+y^{2}=|z|^{2}$\\
    $|zw|^{2}=zw\Bar{zw}=z\Bar{z}w\Bar{w}=|z|^{2}|w|^{2}$\\
    $|\frac{1}{w}|=|\frac{\Bar{w}}{|w|^{2}}|=\frac{|\Bar{w}|}{|w|^{2}}=\frac{1}{|w|}$
    \item $e^{i\theta}=\cos\theta+i\sin\theta$; $e^{i\pi}=-1$; $\Bar{z}=e^{-i\theta}$\\
    Principle value of the argument: $-\pi<\arg(z)\leq\pi$
    \item De Moivres Thm: $wz=\rho re^{i(\varphi+\theta)}$ pf: write $wz$ then compound angle formulae.
    \item $e^{z}=e^{x}(\cos y+i\sin y)$\\
    $|e^{i\theta}|=1$ if $\theta\in\mathbb{R}$; $|e^{z}|=e^{\Re(z)}$\\
    $e^{z}=1\iff z=2\pi ik$, $k\in\mathbb{Z}$; $e^{z+2\pi ik}=e^{z}$
    \item $\cos(x)=\frac{1}{2}(e^{iz}+e^{-iz})$; $\sin(x)=\frac{1}{2i}(e^{iz}-e^{-iz})$
    \item Roots of unity:\\
    Find the exponential form of your $z$.\\
    find the nth root of $|z|$ then your argument is $(arg(z)/n+2k\pi/n)$. Facts:\\
    $\sum^{n}_{j=1}\zeta^{(n)}_{j}=0$; $\prod^{n}_{j=1}\zeta^{(n)}_{j}=(-1)^{n+1}$
    \item Square root:\\
    $a+ib=(x+iy)^{2}=x^{2}-y^{2}+2ixy$;\\
    so $a=x^{2}-y^{2}$, $b=2xy$ $\implies x^{4}-ax^{2}-\frac{1}{4}b^{2}=0$\\
    let $t=x^{2}$ so $t=\frac{a\pm\sqrt{a^{2}+b^{2}}}{2}$. Find $t_{+}$ \& solve.\\
    $x=\pm\sqrt{t_{+}}$ and $y=b/2x$
    \item Curves \& planar regions:\\
    $\Re(e^{i\theta}z)=\cos(\theta)x-\sin(\theta)y=r$\\
    $\implies y=\cot(\theta)x-r\csc(\theta)$
    Similarly $\Im(e^{i\theta}z)=s$ is a line.
    $|z-\omega|=r$: Circle radius $r$ center: $\omega$
%integers and congruences:
    \item \textbf{GCD}: for $a\in\mathbb{Z}$ and $b\in\mathbb{N}$, there exists a unique $q,r\in\mathbb{Z}$ with $0\leq r<b$ such that $a=qb+r$
%Modular Arithmetic:
    \item $a\equiv b(\text{mod }m)$ if $m\mid(a-b)$
    \item To solve: $ax\equiv b(\text{mod }m)$,\\
    Find $d=\gcd(a,m)$\\
    If $d\nmid b$ then no solns\\
    If $d\mid b$ then write $b=b'd$ and write $d=sa+tm$\\
    $x=sb'$ is a soln.
    Let $m'=m/d$. The set of numbers congruent to $sb'(\text{mod }m')$ is the set of all solns
    \item Chinese remainder thm: If $m_{1}$ and $m_{2}$ are s.t. $\gcd(m_{1},m_{2})=1$ then there is a unique $x(\text{mod }m_{1}m_{2})$ that satisfies $x\equiv c_{1}(\text{mod }m_{1})$ and $x\equiv c_{2}(\text{mod }m_{2})$\\
    if $\gcd(m_{1},m_{2})=1$, write $1=sm_{1}+tm_{2}$, then $c_{1}tm_{2}+c_{2}sm_{1}$ is a unique soln modulo $m_{1}m_{2}$
    %Real and complex polynomials
    \item FTA: A degree n complex polyn has exactly n complex roots.
    \item Lagrange interpolation:\\
    $p(x)=\sum^{n+1}_{j=1}p_{j}(x)$ where $p_{j}(x)=y_{j}\prod^{n+1}_{k=1,k\neq j}\frac{x-x_{k}}{x_{j}-x_{k}}$
    %Vectors in 2 & 3 dimensions
    \item Scalar: Number, Vector: Direction \& Magnitude
    \item $\underline{u}\cdot\underline{v}=|\underline{u}||\underline{v}|\cos{\theta}$
    \item $\underline{u}\times\underline{v}=\det\begin{vmatrix}\underline{i} &\underline{j} &\underline{k}\\a_{1} &b_{1} &c_{1}\\a_{2} &b_{2} &c_{2}\end{vmatrix}\\=i\begin{vmatrix}b_{1}&c_{1}\\b_{2}&c_{2}\end{vmatrix}-j\begin{vmatrix}a_{1}&c_{1}\\a_{2}&c_{2}\end{vmatrix}+k\begin{vmatrix}a_{1}&b_{1}\\a_{1}&b_{2}\end{vmatrix}$
    \item Point to line: $(\underline{u}-\underline{r})\cdot\underline{v}=0$
    \item Point to plane: for $\underline{r}\cdot\underline{n}=\underline{p}\cdot\underline{n}$, $(\underline{x}-\underline{x}_{0})\cdot\underline{n}$
\end{itemize}
\end{multicols}
Extra:
%The second year of algebra:
\newpage
\lhead{\color{red} Algebra Pt.2 - Crib Sheet}
\chead{\color{red}Henry Wise}
\rhead{\color{red}Summer Exam EOY}

\begin{multicols}{2}
\begin{itemize}
% Matrices
    \item $i$ is up and down, $j$ is across.\\
    $\mathbb{R}^{m\times n}\implies$ $n$ across, $m$ down.\\
    To multiply a matrix, must have $n_{A}=m_{B}$.
    \item In a matrix product: $c_{ik}=\sum^{n}_{j=1}a_{ij}b_{jk}$; \\Use to prove multiplication.
    \item pf: $(A+B)_{ij}=A_{ij}+B_{ij}=B_{ij}+A_{ij}=(B+A)_{ij}$
    \item$\begin{pmatrix}a &b\\c &d\end{pmatrix}^{T}=\begin{pmatrix}a &c\\b& d\end{pmatrix}$; $A^{T}_{ij}=A_{ji}$\\
    $(AB)^{T}=B^{T}A^{T}$
% General solution of systems of linear equations
    \item Solution of a system of linear equations:\\
    1) Write the problem in matrix notation\\
    2) Construct the "Augmented Matrix"\\
    3) Do "row operations" to get the matrix into Echelon Form
    \item Echelon forms: $\square$: real number, $\star$: non-zero "pivot"\\
    A) $\begin{pmatrix}\star &\square &\square\\ 0 &\star &\square\\0 &0 &\star\end{pmatrix}$
    B) $\begin{pmatrix}\star &\square &\square\\ 0 &\star &\square\\0 &0 &0\end{pmatrix}$\\
    C) $\begin{pmatrix}\star &\square &\square\\ 0 &0 &\star\\0 &0 &0\end{pmatrix}$
    D) $\begin{pmatrix}\star &\square &\square\\ 0 &0 &0\\0 &0 &0\end{pmatrix}$
    \item Reduced Echelon forms:\\
    A) $\begin{pmatrix}\star &0 &0\\ 0 &\star &0\\0 &0 &\star\end{pmatrix}$
    B) $\begin{pmatrix}\star &0 &\square\\ 0 &\star &\square\\0 &0 &0\end{pmatrix}$
    C) $\begin{pmatrix}\star &\square &0\\ 0 &0 &\star\\0 &0 &0\end{pmatrix}$
% The matrix inverse
    \item For $A=\begin{pmatrix}a &b\\c& d\end{pmatrix}$, $A^{-1}=\frac{1}{\det(A)}\begin{pmatrix}d& -b\\-c &a\end{pmatrix}$
% Determinants
    \item $\det(A)=\sum^{n}_{j=1}(-1)^{i+j}a_{ij}\det(A_{ij})\\=\sum^{n}_{i=1}(-1)^{i+j}a_{ij}\det(A_{ij})$
    \item Upper triangular matrix:\\
    $A^{E}=\left(\begin{array}{cccc}
         \mu_{1} &\square &\hdots &\square  \\
         0 &\mu_{2} &\square &\vdots \\
         \vdots &\vdots &\ddots &\square\\
         0 &0 &\hdots &\mu_{n}
    \end{array}\right)$\\
    and, $\det(A^{E})=\mu_{1}\mu_{2}\dots\mu_{n}=\prod^{n}_{i=1}\mu_{i}$\\
    Proof: show result is true for $n=2$, then assume true for $n-1$ with $n-1\geq2$
    \item Eigenvectors \& Eigenvalues example:\\
    $\begin{pmatrix}3 &2\\2 &0\end{pmatrix}\begin{pmatrix}x_{1}\\x_{2}\end{pmatrix}=\lambda\begin{pmatrix}x_{1}\\x_{2}\end{pmatrix}$\\
    The first step is to write:\\
    $\begin{pmatrix}3 &2\\2 &0\end{pmatrix}\begin{pmatrix}x_{1}\\x_{2}\end{pmatrix}=\begin{pmatrix}\lambda &0\\0&\lambda\end{pmatrix}\begin{pmatrix}x_{1}\\x_{2}\end{pmatrix}$\\
    Which is equivalent to:\\
    $\begin{pmatrix}3-\lambda &2\\2&-\lambda\end{pmatrix}\begin{pmatrix}x_{1}\\x_2\end{pmatrix}=\begin{pmatrix}0\\0\end{pmatrix}$\\
    Then we must have:\\
    $\begin{vmatrix}3-\lambda &2\\2&-\lambda\end{vmatrix}=0$\\
    Which gives us $\lambda^{2}-3\lambda-4=0$ so $\lambda=4$ or $\lambda=-1$, two possible eigenvalues.\\
    At $\lambda=4$: $\begin{pmatrix}
        -1 &2\\2&-4
    \end{pmatrix}\begin{pmatrix}x_{1}\\x_{2}\end{pmatrix}=\begin{pmatrix}0\\0\end{pmatrix}$\\
    Which implies $x_{1}=2x_{2}$ so we have $\begin{pmatrix}x_{1}\\x_{2}\end{pmatrix}=c\begin{pmatrix}2\\1\end{pmatrix}$\\
    At $\lambda=-1$: $\begin{pmatrix}4&2\\2&1\end{pmatrix}\begin{pmatrix}x_{1}\\x_{2}\end{pmatrix}=\begin{pmatrix}0\\0\end{pmatrix}$\\
    Which implies $x_{2}=-2x_{1}$ so we have $\begin{pmatrix}x_{1}\\x_{2}\end{pmatrix}=c'\begin{pmatrix}-1\\2\end{pmatrix}$\\
    where $c,c'\neq0$ and so we have our eigenvectors.
\end{itemize} 
\end{multicols}
Extra:
\end{document}
